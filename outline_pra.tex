\documentclass[pra,superscriptaddress,twocolumn,notitlepage,showpacs]{revtex4-1}
%\documentclass[aps,superscriptaddress,twocolumn,notitlepage,showpacs]{revtex4-1}
%\documentclass[aps,superscriptaddress,twocolumn]{nature}
\usepackage{graphicx} 
\usepackage{amsmath}
 
 
\usepackage{lipsum} % dummy text for the MWE
% Put % before of what you want disabled1
% Select what to do with todonotes: 
%\usepackage[disable]{todonotes} % notes not showed
\usepackage[draft]{todonotes}   % notes showed
% Select what to do with command \comment:  

 
\newcommand{\beq}{\begin{equation}}
\newcommand{\enq}{\end{equation}}
\newcommand{\bea}{\begin{eqnarray}}
\newcommand{\ena}{\end{eqnarray}}
\newcommand{\la}{\langle}
\newcommand{\ra}{\rangle}


\begin{document}
\title{Quantum emitter dipole-dipole interactions in nanoplasmonic arrays}
\author{M. Ne\v{c}ada, J.-P. Martikainen, and P. T\"{o}rm\"{a}}
\affiliation{COMP Centre of Excellence, Department of Applied Physics,
Aalto University, P.O. Box 15100, Fi-00076 Aalto, Finland}
\date{\today}
%\pacs{33.80.-b, 73.20.Mf, 42.50.Nn}
\pacs{\dots}
%Photon molecule interactions, 33.80.-b
%Plasmons on surfaces and interfaces, 73.20.Mf
%Superfluorescence, 42.50.Nn
%Bose-Einstein statistics, 05.30.-d
%Bose-Einstein condensates, 67.85.Hj, 67.85.Jk
\begin{abstract}
    \dots
\end{abstract}
\maketitle 


\section*{Introduction}
\begin{itemize}
\item Our focus: the effects of dipole-dipole interactions between the emitters
in the nanoplasmonic array system. What could we see in our systems,
i.e. nanoparticle arrays?
\item Mention Salomon's (slit array; FDTD/Bloch equations) and Delga's (single
sphere; Wubs's frequency domain quantum multiple scattering model)
results. Salomon's parameters are probably bit unrealistic in our
context.
\end{itemize}

\section*{Multiple scattering model}


\subsection*{Model description}
\begin{itemize}
\item Brief model description with references to Wubs and Delga
\item The advantage of this model is that it should give an actual observable
light spectrum.
\end{itemize}

\subsection*{Single nanoparticle case}
\begin{itemize}
\item Custom nanoparticle geometry: description, BEM modelling, classical
scattering properties.\emph{ Timetable item 3. }
\item Results: observable spectra for some realistic parameters (QE concentration,
dipole moment). Far field / near field. \emph{Timetable item 6.}
\end{itemize}

\subsection*{Arrays}
\begin{itemize}
\item Similar as in the previous subsection, but now with multiple nanoparticles
in array. \emph{Timetable item 5.}
\item Results: spectra for array geometries. \emph{Timetable item 6.}
\end{itemize}

\section*{Exactly diagonalised model}

\emph{How to include this into the paper? }We compare the shape of
spectra from the scattering models to the ``simple'' exactly diagonalised
model.
\begin{itemize}
\item Model Hamiltonian
    \begin{multline}
H=\omega b^{\dagger}b+\sum_{i=1}^{K}\epsilon_{i}S_{i}^{z}+\sum_{i=1}^{K}V_{i}\left(b^{\dagger}S_{i}^{-}+S_{i}^{+}b\right)+\\\sum_{i<j=1}^{K}g_{ij}\left(S_{i}^{+}S_{j}^{-}+S_{j}^{+}S_{i}^{-}\right).
\end{multline}


\begin{itemize}
\item Description: only one field mode, no losses, several dipoles, ...
\item How we calculate couplings for real configurations.
\end{itemize}
\item Spectra for $N\lesssim16$, single excitations

\begin{itemize}
\item Coupling strength effects
\item Scale considerations: realistic concentration and dipole moment vs.
thermal energy
\item Randomness effects
\item Model does not show anything about experimental observables in the
real systems.
\item Nearest neighbour approximation shows no qualitative effects at this
scale.
\end{itemize}
\end{itemize}

\section*{Conclusions}
\begin{itemize}
\item What do the results from the two models tell us about how the dipole-dipole
interactions affect the system. At which parameter regions do we get
some interesting effects.\end{itemize}
\bibliographystyle{apsrev}
%\bibliography{./bibli_Plasmons_new}


\begin{thebibliography}{31}
\end{thebibliography}


\end{document}
